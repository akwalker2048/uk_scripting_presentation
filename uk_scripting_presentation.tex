\documentclass{beamer}
\setbeamertemplate{navigation symbols}{}

\usepackage{pslatex}
\usepackage{epsf}
\usepackage{subfigure}

\usetheme{Montpellier}

\beamersetuncovermixins{\opaqueness<1>{25}}{\opaqueness<2->{15}}
\begin{document}
\title{Linux and Scripting}
\subtitle{Making the Hard Stuff Easy}
\author{Andrew Walker}
\date{\today}
\begin{frame}
  \titlepage
\end{frame}

\begin{frame}\frametitle{Table of contents}\tableofcontents
\end{frame}

\section{Robot Demo}
\begin{frame}\frametitle{Introducing Marty the Robot}
  Nothing to see here...this part is just a live demo of a product that:
  \begin{enumerate}
  \item Runs linux
  \item Critical code written in C, C++, and Python
  \item And built with...and running some live...shell scripts...
  \end{enumerate}
\end{frame}

\section{Introduction to Scripting}
\subsection{Before You Get Started}
\begin{frame}\frametitle{Someone Has Already Done This}
  \begin{itemize}
  \item Google is your friend \pause
  \item StackExchange/StackOverflow is your friend \pause
  \item ChatGPT is (well maybe not) your friend \pause
  \item Your frieds are your fireds \pause
  \item Don't be afraid to make mistakes \pause
  \end{itemize}
\end{frame}

\begin{frame}\frametitle{Google Example}
  \begin{figure}[!htb]
    \epsfxsize=0.5\linewidth
    \begin{center}
      \epsffile{./figures/google_is_user_root.eps}
    \end{center}
    \caption{Check User Is Root}\label{fig:google-user-is-root}
  \end{figure}
\end{frame}

\begin{frame}\frametitle{StackExchange/StackOverflow Example}
  \begin{figure}[!htb]
    \epsfxsize=0.25\linewidth
    \begin{center}
      \epsffile{./figures/stackoverflow_is_user_root.eps}
    \end{center}
    \caption{Check User Is Root}\label{fig:stack-user-is-root}
  \end{figure}
\end{frame}

\begin{frame}\frametitle{ChatGPT Example}
  \begin{figure}[!htb]
    \epsfxsize=0.5\linewidth
    \begin{center}
      \epsffile{./figures/chatgpt_am_i_root.eps}
    \end{center}
    \caption{Check User Is Root}\label{fig:chat-user-is-root}
  \end{figure}
\end{frame}

\begin{frame}\frametitle{ChatGPT I Tried To Get Out Of This}
  \begin{figure}[!htb]
    \epsfxsize=0.4\linewidth
    \begin{center}
      \epsffile{./figures/chatgpt_wrote_this_for_me.eps}
    \end{center}
    \caption{Just Do My Presentation For Me}\label{fig:chat-all}
  \end{figure}
\end{frame}

\section{Just Start With The Command Line}
\begin{frame}\frametitle{All Commands Can Be Later Included In Scripts}
  Familiarize with common commands...a few examples:
  \begin{itemize}
  \item history
  \item pwd
  \item whoami
  \item ls -latrh
  \item less
  \item grep
  \item echo
  \item ...
  \end{itemize}

  Some lean themselves more to interactive use...and other scripting...
\end{frame}

\begin{frame}\frametitle{History Example}
    If you did something on the command line that worked and you want to capture it...

  \begin{figure}[!htb]
    \epsfxsize=0.75\linewidth
    \begin{center}
      \epsffile{./figures/history_and_grep_example.eps}
    \end{center}
    \caption{I Just Used A Script For Something Useful}\label{fig:history-grep-example}
  \end{figure}
\end{frame}

\section{And...Do Something With a Script}
\begin{frame}[fragile]\frametitle{Loops}
  Sometimes, you need to loop through files and do something with them:
\begin{verbatim}
#!/bin/sh

for fname in *.png; do
    bname=$(basename -s .png $fname)
    convert $fname eps3:$bname.eps
done
\end{verbatim}
\end{frame}


\section{Closing}
\begin{frame}[fragile]\frametitle{Where Do I Go From Here}
  You can find the code that made this presentation here:
\begin{verbatim}
https://github.com/akwalker2048/uk_scripting_presentation
\end{verbatim}

As long as it doesn't get out of hand...feel free to ask questions at:
\begin{verbatim}
akwalker2048@gmail.com
\end{verbatim}
\end{frame}

\section{If There's Time}
\begin{frame}\frametitle{Other Really Useful Stuff}
  \begin{enumerate}
  \item tmux
  \item git
  \end{enumerate}
\end{frame}

\end{document}
